%%%%%%%%%%%%%%%%%%%%%%%%%%%%%%%%%%%%%%%%%
% Beamer Presentation
% LaTeX Template
% Version 1.0 (10/11/12)
%
% This template has been downloaded from:
% http://www.LaTeXTemplates.com
%
% License:
% CC BY-NC-SA 3.0 (http://creativecommons.org/licenses/by-nc-sa/3.0/)
%
%%%%%%%%%%%%%%%%%%%%%%%%%%%%%%%%%%%%%%%%%

%----------------------------------------------------------------------------------------
%	PACKAGES AND THEMES
%----------------------------------------------------------------------------------------

\documentclass{beamer}

\mode<presentation> {

% The Beamer class comes with a number of default slide themes
% which change the colors and layouts of slides. Below this is a list
% of all the themes, uncomment each in turn to see what they look like.

\usetheme{default}
%\usetheme{AnnArbor}
%\usetheme{Antibes}
%\usetheme{Bergen}
%\usetheme{Berkeley}
%\usetheme{Berlin}
%\usetheme{Boadilla}
%\usetheme{CambridgeUS}
%\usetheme{Copenhagen}
%\usetheme{Darmstadt}
%\usetheme{Dresden}
%\usetheme{Frankfurt}
%\usetheme{Goettingen}
%\usetheme{Hannover}
%\usetheme{Ilmenau}
%\usetheme{JuanLesPins}
%\usetheme{Luebeck}
%\usetheme{Madrid}
%\usetheme{Malmoe}
%\usetheme{Marburg}
%\usetheme{Montpellier}
%\usetheme{PaloAlto}
%\usetheme{Pittsburgh}
%\usetheme{Rochester}
%\usetheme{Singapore}
%\usetheme{Szeged}
%\usetheme{Warsaw}

% As well as themes, the Beamer class has a number of color themes
% for any slide theme. Uncomment each of these in turn to see how it
% changes the colors of your current slide theme.

%\usecolortheme{albatross}
\usecolortheme{beaver}
%\usecolortheme{beetle}
%\usecolortheme{crane}
%\usecolortheme{dolphin}
%\usecolortheme{dove}
%\usecolortheme{fly}
%\usecolortheme{lily}
%\usecolortheme{orchid}
%\usecolortheme{rose}
%\usecolortheme{seagull}
%\usecolortheme{seahorse}
%\usecolortheme{whale}
%\usecolortheme{wolverine}

%\setbeamertemplate{footline} % To remove the footer line in all slides uncomment this line
%\setbeamertemplate{footline}[page number] % To replace the footer line in all slides with a simple slide count uncomment this line

%\setbeamertemplate{navigation symbols}{} % To remove the navigation symbols from the bottom of all slides uncomment this line
}

\usepackage{graphicx} % Allows including images
\usepackage{booktabs} % Allows the use of \toprule, \midrule and \bottomrule in tables

%----------------------------------------------------------------------------------------
%	TITLE PAGE
%----------------------------------------------------------------------------------------

\title{Example document to recreate with beamer in LATEX} % The short title appears at the bottom of every slide, the full title is only on the title page

\author{Florian van Leeuwen} % Your name

\date{\today} % Date, can be changed to a custom date
\institute[UU] % Your institution as it will appear on the bottom of every slide, may be shorthand to save space
{
Markup Languages and Reproducible Programming in Statistics \\ % Your institution for the title page

}
\begin{document}

\begin{frame}
\titlepage % Print the title page as the first slide
\end{frame}

\begin{frame}
\frametitle{Outline} % Table of contents slide, comment this block out to remove it
\tableofcontents
\end{frame}

\section{Working with equations}
\begin{frame}
\frametitle{Working with equations}
We define a set of equations as
\begin{equation}
\label{eq1}
a = b + c^2,
\end{equation}

\begin{equation}
\label{eq2}
a - c^2 = b,
\end{equation}

\begin{equation}
\label{eq3}
left side = right side,
\end{equation}

\begin{equation}
\label{eq4}
left side + something \geq right side,
\end{equation}

for all something $>$ 0.

\end{frame}

\section{Aligning the same equations}
\begin{frame}
\frametitle{Aligning the same equations}
Aligning the equations by the equal sign gives a much better view into the placements of the separate equation components.
\begin{align}
a &= b + c^2, \\
a - c^2 &= b, \\
left side &= right side, \\
left side + something &\geq right side,
\end{align}


\end{frame}

\section{Omit equation numbering}
\begin{frame}
\frametitle{Omit equation numbering}
Alternatively, the equation numbering can be omitted.

\begin{align*}
a &= b + c^2, \\
a - c^2 &= b, \\
left side &= right side, \\
left side + something &\geq right side,
\end{align*}
\end{frame}

\section{Ugly alignment}
\begin{frame}
\frametitle{Ugly alignment}
Some components do not look well, when aligned. Especially equations with different heights and spacing. For example,
\begin{align}
E &= mc^2, \\
m &= \frac{E}{c^2}, \\
c &= \sqrt{\frac{E}{m}}.
\end{align}
Take that into account.

\end{frame}

\section{Discussion}
\begin{frame}
\frametitle{Discussion}
This is where you’d normally give your audience a recap of your talk, where you could discuss e.g. the following
\begin{itemize}
  \item Your main findings
  \item The consequences of your main findings
  \item Things to do
  \item Any other business not currently investigated, but related to your talk
\end{itemize}
\end{frame}

\end{document} 