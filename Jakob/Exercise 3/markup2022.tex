\documentclass[9pt]{beamer}
\usetheme{Rochester}
\usecolortheme{wolverine}

\usepackage[nodisplayskipstretch]{setspace}
\usepackage{amsmath}
\usepackage{hyperref}

\hypersetup{
    colorlinks=true,
}

% title page
\setbeamerfont{title}{size=\large}
\title{Example document to recreate with beamer in \LaTeX}
\author{Jakob Nolte}
\date{FALL 2022 \\
Markup Languages and Reproducible Programming in Statistics}

\begin{document}

\frame{\titlepage}

% table of contents
\section*{Outline}
{
\setbeamerfont{frametitle}{size=\large}
\setbeamertemplate{sections/subsections in toc}[default]
\begin{frame}
    \frametitle{Outline}
    \hypersetup{linkcolor=black}
    \tableofcontents %
\end{frame}
}

% slide 1
\section{Working with equations}
{
\setbeamerfont{frametitle}{size=\large}
\setstretch{0.5}
\begin{frame}
\frametitle{Working with equations}

We define a set of equations as \\
\begin{equation}
    a = b + c^2,
\end{equation}
\begin{equation}
    a - c^2 = b,
\end{equation}
\begin{equation}
    \text{left side} = \text{right side},
\end{equation}
\begin{equation}
    \text{left side + something} \geq \text{right side},
\end{equation}
for all something $>$ 0.

\end{frame}
}

% slide 2
\subsection{Aligning the same equations}
{
\setbeamerfont{frametitle}{size=\large}
\setstretch{0.5}
\begin{frame}
\frametitle{Aligning the same equations}

Aligning the equations by the equal sign gives a much better view into the placements of the separate equation components. \\
\begin{align}
    a &= b + c^2, \\
    a - c^2 &= b, \\
    \text{left side} &= \text{right side}, \\
    \text{left side + something} &\geq \text{right side},
\end{align}
for all something $>$ 0.

\end{frame}
}

% slide 3
\subsection{Omit equation numbering}
{
\setbeamerfont{frametitle}{size=\large}
\setstretch{0.5}
\begin{frame}
\frametitle{Omit equation numbering}

Alternatively, the equation numbering can be omitted. \\
\begin{align}
    a &= b + c^2, \notag\\
    a - c^2 &= b, \notag\\
    \text{left side} &= \text{right side}, \notag\\
    \text{left side + something} &\geq \text{right side}, \notag
\end{align}
for all something $>$ 0.

\end{frame}
}

% slide 4
\subsection{Ugly alignment}
{
\setbeamerfont{frametitle}{size=\large}
\setstretch{0.5}
\begin{frame}
\frametitle{Ugly alignment}

Some components do not look well, when aligned. Especially equations with different heights and spacing. For example, \\
\begin{align}
    E &= mc^2, \\
    m &= \frac{E}{c^2}, \\
    c &= \sqrt{\frac{E}{c^2}}.
\end{align}
Take that into account.

\end{frame}
}

% discussion
\section{Discussion}
{
\setbeamerfont{frametitle}{size=\large}
\setstretch{0.5}
\setbeamertemplate{itemize items}[triangle]
\begin{frame}
\frametitle{Discussion}

This is where you’d normally give your audience a recap of your talk, where you could discuss e.g. the following:
\begin{itemize}
    \item Your main findings
    \item The consequences of your main findings
    \item Things to do
    \item Any other business not currently investigated, but related to your talk
\end{itemize}

\end{frame}
}

\end{document}
