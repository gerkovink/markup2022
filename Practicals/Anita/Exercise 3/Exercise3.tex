%\documentclass[aspectratio=43]{beamer} %4:3 ratio presentatie
\documentclass[aspectratio=169]{beamer} %16:9 ratio presentatie


\usetheme[]{default}
\usecolortheme{beaver}
\usepackage{tikz}
\usepackage{multirow}
\usepackage{xcolor}
\usepackage{graphicx}

\beamertemplatenavigationsymbolsempty %suppress navigation bar
\definecolor{light-gray}{gray}{0.1}

%TITLEPAGE
\title[Example] {Example document to recreate with \texttt{beamer} in \LaTeX}

\author[Your Name]
{
 Your~Name
}

\date[MLRPS]
{\vspace{.5 in}\\ FALL 2022 \\ Markup Languages and Reproducible Programming in Statistics  \vskip6mm}



\begin{document}
\titlepage

\begin{frame}
\frametitle{Outline}
\tableofcontents
\end{frame}

%%%%%%%%%
\section{Working with equations}
\begin{frame}
\frametitle{Working with equations}
    We define a set of equations as
    \begin{equation}
        a=b+c^2,
    \end{equation}
    \begin{equation}
    	a-c^2=b,
    \end{equation}
    \begin{equation}
    	\text{left side} = \text{right side},
    \end{equation}
    \begin{equation}
    	\text{left side} + \text{something} \geq \text{right side},
    \end{equation}
    for all $\text{something} > 0$. 
\end{frame}

%%%%%%%%%%
\subsection{Aligning the same equations}
\begin{frame}
\frametitle{Aligning the same equations}
    Aligning the equations by the equal sign gives a much better view into the placements of the separate equation components. 
    \begin{align}
    	a&=b+c^2,\\
    	a-c^2&=b,\\
    	\text{left side} &= \text{right side},\\
    	\text{left side} + \text{something} & \geq \text{right side},
    \end{align}
\end{frame}

%%%%%%%%%%
\subsection{Omit equation numbering}
\begin{frame}
\frametitle{Omit equation numbering}
    Alternatively, the equation numbering can be omitted. 
    \begin{align*}
    	a&=b+c^2\\
    	a-c^2&=b\\
    	\text{left side} &= \text{right side}\\
    	\text{left side} + \text{something} & \geq \text{right side}
    \end{align*}
\end{frame}

%%%%%%%%%%
\subsection{Ugly alignment}
\begin{frame}
\frametitle{Ugly alignment}
Some components do not look well, when aligned. Especially equations with different heights and spacing. For example, 
\begin{align}
E&=mc^2,\\
m&=\frac{E}{c^2},\\
c&=\sqrt{\frac{E}{m}}.
\end{align}
Take that into account. 
\end{frame}

%%%%%%%%%%
\section{Discussion}
\begin{frame}
\frametitle{Discussion}
This is where you'd normally give your audience a recap of your talk, where you could discuss e.g. the following
\begin{itemize}
\item Your main findings
\item The consequences of your main findings
\item Things to do
\item Any other business not currently investigated, but related to your talk
\end{itemize}
\end{frame}


\end{document}

