\documentclass{beamer} 
\usetheme{default}
\usecolortheme{beaver}
\usepackage{tikz}
\usepackage{multirow}
\usepackage{xcolor}
\usepackage{graphicx}

\beamertemplatenavigationsymbolsempty
\definecolor{light-gray}{gray}{0.1}

\title[Example] {Why Cats Are Superior to Dogs - An Exercise Using \texttt{beamer} in \LaTeX}
\author[Your Name]{E P Swens}
\date[MLRPS]
{\vspace{.5 in}\\ Markup Languages}
\begin{document}
\titlepage

\begin{frame}
\frametitle{Table of Contents}
\tableofcontents
\end{frame}

\section{Proof}

\subsection{Simple Algebraic Proof}
\begin{frame}
\frametitle{Simple Algebraic Proof}
    We define a set of equations as
    \begin{equation}
        \text{cats} \ne \text{dogs},
    \end{equation}
    \begin{equation}
    	\text{cats} > \text{dogs},
    \end{equation}
    \begin{equation}
    \text{cats} = \text{cats},
    \end{equation}
    \begin{equation}
    	\text{dogs} \ngtr \text{cats},
    \end{equation}
    $\forall \; \text{cats} \subseteq \mathbb U$. 
\end{frame}

\subsection{Prettier Algebraic Proof}
\begin{frame}
\frametitle{Prettier Algebraic Proof}
    Aligning the equations by the equal sign provides a prettier proof. 
    \begin{align}
    	\text{cats} \ne \text{dogs},\\
    	\text{cats} > \text{dogs}, \\
    	\text{cats} = \text{cats},\\
    	\text{dogs} \ngtr \text{cats}.
    \end{align}
\end{frame}

\subsection{Uglier Algebraic Proof}
\begin{frame}
\frametitle{Uglier Algebraic Proof}
Some proofs are better left unaligned
\begin{align}
\text{cats} &= \text{cats},\\
\text{cats} &=\frac{\text{cats}^2}{\text{cats}},\\
\text{cats} &=\sqrt{\frac{\text{cats}^3}{\text{cats}}}.
\end{align}
\end{frame}

\section{Concluding}
\begin{frame}
\frametitle{Concluding}
This presentation showed that
\begin{itemize}
\item cats are superior to dogs
\end{itemize}
\end{frame}


\end{document}

