\documentclass[11pt, aspectratio=169]{beamer}
\usetheme{default}
\usecolortheme{beaver}
\beamertemplatenavigationsymbolsempty

\usepackage{amsmath}

\title{Example document to recreate with \texttt{beamer} in \LaTeX}
\author{Christine Hedde - von Westernhagen}
\date{}


\begin{document}

\begin{frame}[plain]
  \titlepage
  \vspace{\fill}
  \centering FALL 2019\\Markup Languages and Reproducible Programming in Statistics
\end{frame}


\begin{frame}{Outline}
    \hyperlink{working}{Working with equations} \\
    \hspace{5mm} \hyperlink{aligning}{Aligning the same equations} \\
    \hspace{5mm} \hyperlink{omit}{Omit equation numbering} \\
    \hspace{5mm} \hyperlink{ugly}{Ugly alignment} \\
    \vspace{10mm}
    \hyperlink{discussion}{Discussion}
\end{frame}


\begin{frame}[label=working]{Working with equations}
    We define a set of equations as
    \begin{equation} a = b + c^2, \end{equation}
    \begin{equation} a - c^2 = b, \end{equation}
    \begin{equation} \text{left side = right side,} \end{equation}
    \begin{equation} \text{left side + something} \geq \text{right side,} \end{equation}
    for all something $> 0$.
\end{frame}


\begin{frame}[label=aligning]{Aligning the same equations}
    Aligning the equations by the equal sign gives a much better view into the placements of the separate equation components.
    \begin{align}
        a &= b + c^2, \\
        a - c^2 &= b, \\
        \text{left side} &= \text{right side,} \\
        \text{left side + something} &\geq \text{right side,}
    \end{align}
\end{frame}


\begin{frame}[label=omit]{Omit equation numbering}
    Alternatively, the equation numbering can be omitted.
    \begin{align*}
        a &= b + c^2 \\
        a - c^2 &= b \\
        \text{left side} &= \text{right side} \\
        \text{left side + something} &\geq \text{right side}
    \end{align*}
\end{frame}


\begin{frame}[label=ugly]{Ugly alignment}
    Some components do not look well, when aligned. Especially equations with different heights and spacing. For example,
    \begin{align}
        E &= mc^2, \\
        m &= \frac{E}{c^2}, \\
        c &= \sqrt{\frac{E}{m}}.
    \end{align}
    Take that into account.
\end{frame}


\begin{frame}[label=discussion]{Discussion}
    This is where you’d normally give your audience a recap of your talk, where you could discuss e.g. the following
    \begin{itemize}
        \item Your main findings
        \item The consequences of your main findings
        \item Things to do
        \item Any other business not currently investigated, but related to your talk
    \end{itemize}
\end{frame}


\end{document}