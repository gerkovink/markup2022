%%%%%%%%%%%%%%%%%%%%%%%%%%%%%%%%%%%%%%%%%
% Beamer Presentation
% LaTeX Template
% Version 1.0 (10/11/12)
%
% This template has been downloaded from:
% http://www.LaTeXTemplates.com
%
% License:
% CC BY-NC-SA 3.0 (http://creativecommons.org/licenses/by-nc-sa/3.0/)
%
%%%%%%%%%%%%%%%%%%%%%%%%%%%%%%%%%%%%%%%%%

%----------------------------------------------------------------------------------------
%	PACKAGES AND THEMES
%----------------------------------------------------------------------------------------

\documentclass{beamer}
\mode<presentation> {
\usecolortheme{beaver}
}

\usepackage{graphicx} % Allows including images
\usepackage{booktabs} % Allows the use of \toprule, \midrule and \bottomrule in tables
\usepackage{amsmath}


\setbeamerfont{title}{size=\large}
\title{Example document to recreate with beamer in \LaTeX}
\author{Maaike Walraad}
\date{}

\begin{document}


\begin{frame}
  \titlepage  
  %\vspace{}
  \centering FALL 2022 \\ Markup Languages and Reproducible Programming in Statistics\par
\end{frame}

\begin{frame}
    \frametitle{Outline} 
    % \tableofcontents[currentsections] % in case of using \section or \subsection
    Working with equations \\
    \hspace{4.5mm} Aligning the same equations \\
    \hspace{4.5mm} Omit equation numbering \\
    \hspace{4.5mm} Ugly alignment \\
    \vspace{10mm}
    Discussion
\end{frame}

\begin{frame}
    \frametitle{Working with equations}
    \vspace{2mm}
    We define a set of equations as:
    
    \begin{equation}
    a = b + c^2,
    \end{equation}
    \begin{equation}
    a - c^2 = b,
    \end{equation}
    \begin{equation}
    \text{left side = right side,}
    \end{equation}
    \begin{equation}
    \text{left side + something} \geq \text{right side,}
    \end{equation}
    
    for all something $>{0}$.
\end{frame}

\begin{frame}
    \frametitle{Aligning the same equations}
    Aligning the equations by the equal sign gives a much better view into the placements
of the separate equation components.
    
    %\begin{equation}
    \begin{align} 
    a &= b + c^2, \\ 
    a - c^2 &= b, \\ 
    \text{left side} &= \text{right side,} \\ 
    \text{left side + something} &\geq \text{right side,}
    \end{align} 
    %\end{equation}
\end{frame}

\begin{frame}
    \frametitle{Omit equation numbering}
    Alternatively, the equation numbering can be omitted. % by using * 
    
    \begin{align*} 
    a &= b + c^2 \\ 
    a - c^2 &= b \\ 
    \text{left side} &= \text{right side} \\ 
    \text{left side + something} &\geq \text{right side}
    \end{align*}
\end{frame}


\begin{frame}
    \frametitle{Ugly alignment}
    Some components do not look well, when aligned. Especially equations with different
heights and spacing. For example,

    \begin{align} 
    E &= mc^2, \\ 
    m &= \frac{E}{c^2}, \\ 
    c &= \sqrt{E}{m}. 
    \end{align} 

    Take that into account.
\end{frame}

\begin{frame}
    \frametitle{Discussion}
    This is where you’d normally give your audience a recap of your talk, where you could
discuss e.g. the following
\begin{itemize}
    \item Your main findings
    \item The consequences of your main findings
    \item Things to do
    \item Any other business not currently investigated, but related to your talk
\end{itemize}
\end{frame}


\end{document} 